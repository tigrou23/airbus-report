\documentclass[11pt]{article}

\usepackage[a4paper,margin=2.2cm]{geometry}
\usepackage{graphicx}
\usepackage{float}
\usepackage{booktabs}
\usepackage{caption}
\usepackage{subcaption}
\usepackage{hyperref}
\usepackage{xcolor}
\usepackage{listings}

\hypersetup{
  colorlinks=true,
  linkcolor=blue,
  urlcolor=blue,
  citecolor=blue
}

\lstdefinestyle{py}{
  language=Python,
  basicstyle=\ttfamily\small,
  keywordstyle=\color{blue},
  commentstyle=\color{gray},
  stringstyle=\color{teal},
  showstringspaces=false,
  breaklines=true,
  frame=single,
  rulecolor=\color{black!20},
  tabsize=4
}
\lstset{style=py}

\title{Python in Finance\\Quantitative Trading Strategy Analysis (Airbus)}
\author{Baptiste Gillot}
\date{\today}

\begin{document}
\maketitle

\begin{abstract}
This report presents a quantitative analysis workflow implemented in Python, applied to Airbus (ticker \texttt{AIR.PA}). 
The deliverable follows the same structure as the provided reference report: (i) a single-asset technical-indicator analysis and strategy backtest, and (ii) a multi-asset portfolio analysis with optimization. 
All interpretations are intentionally omitted; figures produced by the notebook are referenced throughout.
\end{abstract}

\section{Quantitative Trading Strategy Analysis for Airbus Stock}

\subsection{Analysis Objective}
Use historical price data for Airbus to compute technical indicators and build a simple rule-based trading strategy for backtesting.

\subsection{Analysis Method}
The analysis uses:
\begin{itemize}
  \item \texttt{yfinance} for data collection,
  \item \texttt{pandas} and \texttt{numpy} for data processing,
  \item \texttt{matplotlib} for visualization,
  \item a simple RSI-threshold trading rule for backtesting.
\end{itemize}

\subsection{Core Code Example}

\subsubsection{Step 1: Data Acquisition and Cleaning}
The notebook downloads \texttt{AIR.PA} historical data and removes rows containing missing values.

\begin{lstlisting}
import yfinance as yf
import pandas as pd

SYMBOL = "AIR.PA"
START_DATE = "2014-01-01"
END_DATE   = "2024-01-01"

stock_raw = yf.download(SYMBOL, start=START_DATE, end=END_DATE, progress=False)
stock = stock_raw.dropna(how="any")
\end{lstlisting}

\subsubsection{Step 2: Moving Averages}
The notebook computes and plots two moving averages based on Adjusted Close: MA5 and MA20.

\begin{lstlisting}
stock["MA5"]  = stock["Adj Close"].rolling(window=5).mean()
stock["MA20"] = stock["Adj Close"].rolling(window=20).mean()
\end{lstlisting}

\begin{figure}[H]
  \centering
  \includegraphics[width=0.92\textwidth]{figures/airbus_moving_average.png}
  \caption{Airbus stock price and moving averages (MA5, MA20).}
  \label{fig:ma}
\end{figure}

\subsubsection{Step 3: Relative Strength Index (RSI-14)}
The notebook computes the RSI with a 14-day rolling window.

\begin{lstlisting}
delta = price.diff()
gain  = delta.clip(lower=0)
loss  = -delta.clip(upper=0)

avg_gain = gain.rolling(window=14).mean()
avg_loss = loss.rolling(window=14).mean()

rs  = avg_gain / avg_loss
rsi = 100 - (100 / (1 + rs))
\end{lstlisting}

\begin{figure}[H]
  \centering
  \includegraphics[width=0.92\textwidth]{figures/airbus_rsi.png}
  \caption{Airbus RSI-14.}
  \label{fig:rsi}
\end{figure}

\subsubsection{Step 4: Strategy Backtesting}
The notebook applies a discrete trading rule based on RSI thresholds:
\begin{itemize}
  \item Buy when RSI $<$ 30,
  \item Sell when RSI $>$ 70.
\end{itemize}
The portfolio value is tracked over time (fees and taxes are ignored in the model).

\begin{figure}[H]
  \centering
  \includegraphics[width=0.92\textwidth]{figures/airbus_rsi_strategy_portfolio_value.png}
  \caption{Backtest portfolio value for the RSI threshold strategy on Airbus.}
  \label{fig:bt}
\end{figure}

\section{Investment Portfolio Optimization Strategy Analysis}

\subsection{Analysis Objective}
Compute returns/risk metrics for a small multi-asset portfolio (including Airbus) and obtain an allocation that maximizes the Sharpe ratio.

\subsection{Analysis Method}
The notebook:
\begin{itemize}
  \item downloads Adjusted Close prices for a chosen set of assets,
  \item computes daily returns, cumulative returns, annualized metrics, and correlations,
  \item performs Sharpe-ratio optimization with constraints $\sum w_i = 1$ and $0 \le w_i \le 1$ using \texttt{scipy.optimize}.
\end{itemize}

\subsection{Portfolio Assets}
The default asset set in the notebook is:
\begin{center}
\texttt{AIR.PA} (Airbus), \texttt{BA} (Boeing), \texttt{SAF.PA} (Safran), \texttt{GLD} (Gold ETF).
\end{center}
You can change this list directly in the notebook.

\subsection{Key Figures Produced by the Notebook}

\begin{figure}[H]
  \centering
  \includegraphics[width=0.92\textwidth]{figures/portfolio_cumulative_returns.png}
  \caption{Cumulative returns of the portfolio assets.}
  \label{fig:cum}
\end{figure}

\begin{figure}[H]
  \centering
  \begin{subfigure}{0.49\textwidth}
    \centering
    \includegraphics[width=\textwidth]{figures/portfolio_annualized_returns.png}
    \caption{Annualized returns.}
  \end{subfigure}
  \begin{subfigure}{0.49\textwidth}
    \centering
    \includegraphics[width=\textwidth]{figures/portfolio_annualized_volatility.png}
    \caption{Annualized volatility (risk).}
  \end{subfigure}
  \caption{Annualized metrics for each asset.}
  \label{fig:ann}
\end{figure}

\begin{figure}[H]
  \centering
  \includegraphics[width=0.70\textwidth]{figures/portfolio_correlation_matrix.png}
  \caption{Correlation matrix of daily returns.}
  \label{fig:corr}
\end{figure}

\begin{figure}[H]
  \centering
  \includegraphics[width=0.80\textwidth]{figures/portfolio_optimal_weights.png}
  \caption{Optimized weights (max Sharpe) under long-only constraints.}
  \label{fig:w}
\end{figure}

\begin{figure}[H]
  \centering
  \includegraphics[width=0.92\textwidth]{figures/portfolio_equal_vs_optimal.png}
  \caption{Comparison of cumulative returns: equal-weight vs optimized (max Sharpe) portfolio.}
  \label{fig:cmp}
\end{figure}

\appendix
\section{Reproducibility Notes}
\begin{itemize}
  \item Run the notebook first to generate all figures under \texttt{figures/}.
  \item Then compile this \LaTeX{} file. The report includes the generated images by relative path.
  \item The notebook also exports:
    \begin{itemize}
      \item \texttt{figures/airbus\_stock\_data.xlsx},
      \item \texttt{figures/portfolio\_asset\_metrics.csv},
      \item \texttt{figures/portfolio\_portfolio\_metrics.csv}.
    \end{itemize}
\end{itemize}

\end{document}